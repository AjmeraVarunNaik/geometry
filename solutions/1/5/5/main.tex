\iffalse
\documentclass[]{article}
\usepackage{amsfonts, amsmath, amssymb}
\usepackage[]{setspace}
\doublespacing

\begin{document}
\providecommand{\pr}[1]{\ensuremath{\Pr\left(#1\right)}}
\providecommand{\prt}[2]{\ensuremath{p_{#1}^{\left(#2\right)} }}        % own macro for this question
\providecommand{\qfunc}[1]{\ensuremath{Q\left(#1\right)}}
\providecommand{\sbrak}[1]{\ensuremath{{}\left[#1\right]}}
\providecommand{\lsbrak}[1]{\ensuremath{{}\left[#1\right.}}
\providecommand{\rsbrak}[1]{\ensuremath{{}\left.#1\right]}}
\providecommand{\brak}[1]{\ensuremath{\left(#1\right)}}
\providecommand{\lbrak}[1]{\ensuremath{\left(#1\right.}}
\providecommand{\rbrak}[1]{\ensuremath{\left.#1\right)}}
\providecommand{\cbrak}[1]{\ensuremath{\left\{#1\right\}}}
\providecommand{\lcbrak}[1]{\ensuremath{\left\{#1\right.}}
\providecommand{\rcbrak}[1]{\ensuremath{\left.#1\right\}}}
\newcommand{\sgn}{\mathop{\mathrm{sgn}}}
\providecommand{\abs}[1]{\left\vert#1\right\vert}
\providecommand{\res}[1]{\Res\displaylimits_{#1}} 
\providecommand{\norm}[1]{\left\lVert#1\right\rVert}
%\providecommand{\norm}[1]{\lVert#1\rVert}
\providecommand{\mtx}[1]{\mathbf{#1}}
\providecommand{\mean}[1]{E\left[ #1 \right]}
\providecommand{\cond}[2]{#1\middle|#2}
\providecommand{\fourier}{\overset{\mathcal{F}}{ \rightleftharpoons}}
\newenvironment{amatrix}[1]{%
  \left(\begin{array}{@{}*{#1}{c}|c@{}}
}{%
  \end{array}\right)
}
%\providecommand{\hilbert}{\overset{\mathcal{H}}{ \rightleftharpoons}}
%\providecommand{\system}{\overset{\mathcal{H}}{ \longleftrightarrow}}
	%\newcommand{\solution}[2]{\textbf{Solution:}{#1}}
\newcommand{\solution}{\noindent \textbf{Solution: }}
\newcommand{\cosec}{\,\text{cosec}\,}
\providecommand{\dec}[2]{\ensuremath{\overset{#1}{\underset{#2}{\gtrless}}}}
\newcommand{\myvec}[1]{\ensuremath{\begin{pmatrix}#1\end{pmatrix}}}
\newcommand{\mydet}[1]{\ensuremath{\begin{vmatrix}#1\end{vmatrix}}}
\newcommand{\myaugvec}[2]{\ensuremath{\begin{amatrix}{#1}#2\end{amatrix}}}
\providecommand{\rank}{\text{rank}}
\providecommand{\pr}[1]{\ensuremath{\Pr\left(#1\right)}}
\providecommand{\qfunc}[1]{\ensuremath{Q\left(#1\right)}}
	\newcommand*{\permcomb}[4][0mu]{{{}^{#3}\mkern#1#2_{#4}}}
\newcommand*{\perm}[1][-3mu]{\permcomb[#1]{P}}
\newcommand*{\comb}[1][-1mu]{\permcomb[#1]{C}}
\providecommand{\qfunc}[1]{\ensuremath{Q\left(#1\right)}}
\providecommand{\gauss}[2]{\mathcal{N}\ensuremath{\left(#1,#2\right)}}
\providecommand{\diff}[2]{\ensuremath{\frac{d{#1}}{d{#2}}}}
\providecommand{\myceil}[1]{\left \lceil #1 \right \rceil }
\newcommand\figref{Fig.~\ref}
\newcommand\tabref{Table~\ref}
\newcommand{\sinc}{\,\text{sinc}\,}
\newcommand{\rect}{\,\text{rect}\,}
%%
%	%\newcommand{\solution}[2]{\textbf{Solution:}{#1}}
%\newcommand{\solution}{\noindent \textbf{Solution: }}
%\newcommand{\cosec}{\,\text{cosec}\,}
%\numberwithin{equation}{section}
%\numberwithin{equation}{subsection}
%\numberwithin{problem}{section}
%\numberwithin{definition}{section}
%\makeatletter
%\@addtoreset{figure}{problem}
%\makeatother

%\let\StandardTheFigure\thefigure
\let\vec\mathbf

\fi
1.5.5 Repeat the above excercise for the sides $AB$ and $AC$.

\solution  
We know the value of $\vec{I}$ is
\begin{align}
\vec{I} &= \frac{1}{\sqrt{37} + 4 + \sqrt{61}} \myvec{\sqrt{61} - 16 - 3\sqrt{37}\\ -\sqrt{61} + 24 - 5\sqrt{37}}
\end{align}
from the problem 1.5.2 .
\begin{enumerate}
\item {The equation of $AB$ is:
\begin{align}
\myvec{7&5}\vec{x} - 2=0
\end{align}

Let $r_1$ be the distance between $\vec{I}$ and $AB$, then
\begin{align}
r_1 &= \frac{\abs{\myvec{7&5} \vec{I} - 2}}{\norm{\myvec{7 \\ 5}}} \\
&= \frac{\abs{\frac{1}{\sqrt{37} + 4 + \sqrt{61}} \myvec{7&5} \myvec{\sqrt{61} - 16 - 3\sqrt{37}\\ -\sqrt{61} + 24 - 5\sqrt{37}} - 2}}{\sqrt{{7}^2 + {5}^2}} \\
&= \frac{\frac{2\sqrt{61} - 46\sqrt{37} + 8}{\sqrt{37} + 4 + \sqrt{61}} - 2}{\sqrt{74}} \\
&= \frac{48\sqrt{37}}{\sqrt{74} {(\sqrt{37} + 4 + \sqrt{61}})} \\
&= \frac{48}{\sqrt{2}(\sqrt{37} + 4 + \sqrt{61})}\\
&= \frac{24\sqrt{2}}{\sqrt{37}+4+\sqrt{61}}\\
&= 1.8969                                        
\end{align}}
\item{Similarly, the equation of $AC$ is
\begin{align}
\myvec{4&-4}\vec{x} - 8=0
\end{align}

Let $r_2$ be the distance between $\vec{I}$ and $AC$, then
\begin{align}
r_2 &= \frac{\abs{\myvec{4&-4} \vec{I} - 8}}{\norm{\myvec{4 \\ -4}}} \\
&= \frac{\abs{\frac{1}{\sqrt{37} + 4 + \sqrt{61}} \myvec{4&-4} \myvec{\sqrt{61} - 16 - 3\sqrt{37}\\ -\sqrt{61} + 24 - 5\sqrt{37}} - 8}}{\sqrt{{4}^2 + {(-4)}^2}} \\
&= \frac{\abs{\frac{8\sqrt{61} + 8\sqrt{37} - 160}{\sqrt{37} + 4 + \sqrt{61}} - 8}}{4\sqrt{2}} \\
&= \frac{192}{4\sqrt{2} {(\sqrt{37} + 4 + \sqrt{61}})} \\
&= \frac{48}{\sqrt{2}(\sqrt{37} + 4 + \sqrt{61})}\\
&= \frac{24\sqrt{2}}{\sqrt{37}+4+\sqrt{61}}\\
&= 1.8969       
\end{align} }
\end{enumerate}

